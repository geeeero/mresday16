\documentclass{beamer}

% opacity bugfix: see http://tug.org/pipermail/pdftex/2007-December/007480.html
\pdfpageattr {/Group << /S /Transparency /I true /CS /DeviceRGB>>}

\usepackage[utf8]{inputenc}
\usepackage[OT1]{fontenc}

\usepackage{tikz}
\usetikzlibrary{%
   arrows,%
   calc,%
   fit,%
   patterns,%
   plotmarks,%
   shapes.geometric,%
   shapes.misc,%
   shapes.symbols,%
   shapes.arrows,%
   shapes.callouts,%
   shapes.multipart,%
   shapes.gates.logic.US,%
   shapes.gates.logic.IEC,%
   er,%
   automata,%
   backgrounds,%
   chains,%
   topaths,%
   trees,%
   petri,%
   mindmap,%
   matrix,%
   calendar,%
   folding,%
   fadings,%
   through,%
   patterns,%
   positioning,%
   scopes,%
   decorations.fractals,%
   decorations.shapes,%
   decorations.text,%
   decorations.pathmorphing,%
   decorations.pathreplacing,%
   decorations.footprints,%
   decorations.markings,%
   shadows}
\usepackage{animate}
\usepackage{amssymb, amsmath, amsfonts, enumerate}
%\usepackage{bbold}
\newcommand\hmmax{0}
\usepackage{bm}
%\usepackage{dsfont}
\usepackage{pxfonts}
\usepackage{xcolor}
\usepackage{url}

\usepackage[backend=bibtex,style=authoryear,dashed=false]{biblatex}
\addbibresource{refs.bib}
%\renewcommand{\bibfont}{\normalfont\scriptsize}
\setlength{\bibhang}{3ex}

\usepackage{hyperref}

\usetheme[official=false,department=none]{tue2008}
%\usefonttheme{default}


%\usetheme[secheader]{Boadilla}
%\setbeamercovered{transparent}
%\setbeamercovered{invisible}
%\setbeamertemplate{navigation symbols}{}
%\setbeamertemplate{bibliography item}[text] % numbered references
%\useoutertheme{infolines}
%\setbeamertemplate{headline}{}
%\setbeamertemplate{footline}{\hspace*{5mm}\hfill\insertframenumber\hspace*{5mm}\vspace{3mm}}
%\setbeamercolor{alerted text}{fg=orange!80!black}

% ------------------------------------------------------------------------------

\newcommand{\vcenterbox}[1]{\ensuremath{\vcenter{\hbox{#1}}}}
%
\newcommand{\reals}{\mathbb{R}}
\newcommand{\posreals}{\reals_{>0}}
\newcommand{\posrealszero}{\reals_{\ge 0}}
\newcommand{\naturals}{\mathbb{N}}

\newcommand{\dd}{\,\mathrm{d}}

\newcommand{\mbf}[1]{\mathbf{#1}}
\newcommand{\bs}[1]{\boldsymbol{#1}}
\renewcommand{\vec}[1]{{\bm#1}}

\newcommand{\uz}{^{(0)}} % upper zero
\newcommand{\un}{^{(n)}} % upper n
\newcommand{\ui}{^{(i)}} % upper i
\newcommand{\uell}{^{(\ell)}} % upper ell

\newcommand{\ul}[1]{\underline{#1}}
\newcommand{\ol}[1]{\overline{#1}}

\newcommand{\Rsys}{R_\text{sys}}
\newcommand{\lRsys}{\ul{R}_\text{sys}}
\newcommand{\uRsys}{\ol{R}_\text{sys}}

\newcommand{\Fsys}{F_\text{sys}}
\newcommand{\lFsys}{\ul{F}_\text{sys}}
\newcommand{\uFsys}{\ol{F}_\text{sys}}

\def\Tsys{T_\text{sys}}

\def\tmax{t_\text{max}}
\def\tnow{t_\text{now}}
\def\tpnow{t^+_\text{now}}

\newcommand{\ptk}{p^k_t}

\newcommand{\E}{\operatorname{E}}
\newcommand{\V}{\operatorname{Var}}
\newcommand{\sd}{\operatorname{sd}}

\newcommand{\wei}{\operatorname{Wei}} % Weibull Distribution
\newcommand{\ig}{\operatorname{IG}}   % Inverse Gamma Distribution
\newcommand{\ber}{\operatorname{Bernoulli}} 
\newcommand{\bin}{\operatorname{Binomial}}
\newcommand{\be}{\operatorname{Beta}} 
\newcommand{\bebin}{\operatorname{Beta-binomial}}
\newcommand{\norm}{\operatorname{N}}

\def\then{{\structure{$\rule[0.35ex]{2ex}{0.5ex}\!\!\!\blacktriangleright$}}}
\def\play{{\structure{$\blacktriangleright$}}}
\def\gplus{{\structure{\rule[0.45ex]{1.4ex}{0.4ex}\hspace{-0.9ex}\rule[0.0ex]{0.4ex}{1.3ex}\hspace{0.5ex}}}}
\def\gminus{{\structure{\rule[0.45ex]{1.4ex}{0.4ex}}}}

\input{nydefs.tex}
\def\blau#1{{\color{tuegreen}#1}}
\def\rot#1{{\color{tuered}#1}}
\def\gruen#1{{\color{tueblue}#1}}

\def\yzr{\rot{\yz}}
\def\ynr{\rot{\yn}}
\def\byzr{\rot{\byz}}
\def\bynr{\rot{\byn}}
\def\yzor{\rot{y\uz_1}}
\def\yzjr{\rot{y\uz_j}}
\def\yzkr{\rot{y\uz_k}}
\def\yzlr{\rot{\yzl}}
\def\yzur{\rot{\yzu}}
\def\ynjr{\rot{y\un_j}}
\def\ynlr{\rot{\ynl}}
\def\ynur{\rot{\ynu}}
\def\yzjlr#1{\rot{\ul{y}\uz_#1}}
\def\yzjur#1{\rot{\ol{y}\uz_#1}}


\def\nzg{\gruen{\nz}}
\def\nng{\gruen{\nn}}
\def\nzlg{\gruen{\nzl}}
\def\nzug{\gruen{\nzu}}
\def\nnlg{\gruen{\nnl}}
\def\nnug{\gruen{\nnu}}

\def\psib{\blau{\psi}}
\def\bpsib{\blau{{b}(\psi)}}


% ------------ shading start
\newsavebox{\tempbox}
\newcommand\leftrightshading[3]{%
  \begin{tikzfadingfrompicture}[name=inputtext]
    \node [text=white] {#1};
  \end{tikzfadingfrompicture}
  \begin{lrbox}{\tempbox}%
    \begin{tikzpicture}
      \node [text=white,inner sep=0pt,outer sep=0pt] (textnode) {#1};
      \shade[path fading=inputtext,fit fading=false,left color=#2,right color=#3]
      (textnode.south west) rectangle (textnode.north east);
    \end{tikzpicture}%
  \end{lrbox}
  % Now we use the fading in another picture:
  \usebox\tempbox{}%
}
% ------------ shading end


%\def\PZc{\mathrm I\!\Pi\uz}
\def\PZc{\leftrightshading{$\mathrm I\!\Pi\uz$}{blue}{red}}
%\def\PNc{\PN}
\def\PNc{\leftrightshading{$\mathrm I\!\Pi\un$}{blue}{red}}



%%\def\blau#1{{\color{lmugreen2}#1}}
%\def\rot#1{{\color{red}#1}}
%\def\gruen#1{{\color{blue}#1}}

\newcommand{\x}{\vec{x}}

\def\tnow{t_\text{now}}
\def\tpnow{t^+_\text{now}}

\newcommand{\comp}[1]{\raisebox{-1mm}{\tikz{\node[%type1
rectangle,rounded corners=0mm,draw,fill=tuepmsgreen!70,thick,inner sep=0pt,minimum size=4mm]{#1};}}}

\newcommand{\cyansec}[1]{\textcolor{tuecyan}{\large\bf #1}}
\newcommand{\cyanalert}[1]{\textcolor{tuecyan}{#1}}
\newcommand{\bluesec}[1]{\textcolor{tueblue}{\large\bf #1}}
\newcommand{\bluealert}[1]{\textcolor{tueblue}{#1}}

\setbeamertemplate{itemize item}{\tiny\raise1.5pt\hbox{\color{tueblue}$\blacktriangleright$}}

% ------------------------------------------------------------------------------


\title{Nonparametric System Reliability\\ Combining Expert Knowledge and Data}

\author{\ul{Gero Walter}\inst{1}, Louis Aslett\inst{2}, Frank Coolen\inst{3}}
\institute{ \inst{1} Eindhoven University of Technology, Eindhoven, NL\\ 
            \inst{2} University of Oxford, Oxford, UK\\
            \inst{3} Durham University, Durham, UK \\[2ex]
            \url{g.m.walter@tue.nl} \\[2ex]
            \includegraphics[height=9mm]{logos/tuelogo} \quad 
            \includegraphics[height=9mm]{logos/university_of_oxford} \quad
            \includegraphics[height=9mm]{logos/logounidurham-large} \quad
            \includegraphics[height=9mm]{logos/dinalog-hp} }
\date{2016-02-04}

\begin{document}

\frame{
\titlepage
}

\addtocounter{framenumber}{-1} 

\begin{frame}{System Reliability}

\begin{columns}
\begin{column}{0.3\textwidth}
\begin{tikzpicture}
[type1/.style={rectangle,draw,fill=tuepmsgreen!70,very thick,inner sep=0pt,minimum size=6mm},
 type2/.style={rectangle,draw,fill=tuepmsgreen!70,very thick,inner sep=0pt,minimum size=6mm},
 type3/.style={rectangle,draw,fill=tuepmsgreen!70,very thick,inner sep=0pt,minimum size=6mm},
 cross/.style={cross out,draw=red,very thick,minimum width=7mm, minimum height=5mm},
 hv path/.style={thick, to path={-| (\tikztotarget)}},
 vh path/.style={thick, to path={|- (\tikztotarget)}}]
\begin{scope}[scale=0.75]
\node[type1] (T1-2) at ( 1.2, 1.1) {1};
\node[type1] (T1-3) at ( 1.2,-1.1) {1};
\node[type1] (T1-5) at ( 2.8, 1.1) {1};
\node[type1] (T1-6) at ( 2.8,-1.1) {1};
\node[type2] (T2)   at ( 2.0, 0)   {2};
\node[type3] (T3)   at ( 4.3, 0)   {3};
\coordinate (start) at (0  ,0);
\coordinate (end)   at (4.9,0);
\coordinate (bista) at (0.4,0);
\coordinate (biend) at (3.6,0);
\path (bista)     edge[hv path] (start)
                  edge[vh path] (T1-2.west)
                  edge[vh path] (T1-3.west)
      (T1-2.east) edge[hv path] (T2.north)
      (T1-3.east) edge[hv path] (T2.south)
      (T2.north)  edge[vh path] (T1-5.west)
      (T2.south)  edge[vh path] (T1-6.west)
      (biend)     edge[vh path] (T1-5.east)
                  edge[vh path] (T1-6.east)
                  edge[hv path] (T3.west)
      (T3.east)   edge[hv path] (end);
\end{scope}
\end{tikzpicture}
\end{column}
\begin{column}{0.7\textwidth}
We want to learn about the system reliability $\Rsys(t) = P(\Tsys > t)$ based on
\begin{enumerate}
\item[\play]<2-> component test data:
 \begin{itemize}
 \item[] $n_k$ failure times for components of type $k$, $k = 1, \ldots, K$
 \end{itemize}
\item[\play]<3-> cautious assumptions\\ on component reliability:
 \begin{itemize}
 \item[] expert information,\\ e.g.\ from maintenance managers and staff
 \end{itemize}
\end{enumerate}
\end{column}
\end{columns}
\begin{center}
\uncover<4->{\cyanalert{How to combine these two information sources?}}
\end{center}
\end{frame}

\begin{frame}{Bayesian Inference}
\vspace*{-2ex}
\begin{align*}
\begin{array}{ccccl}
\uncover<1->{\text{expert info}        & + & \text{data}                & \to & \text{complete picture} \\[1.5ex]}
\uncover<2->{\text{prior distribution} & + & \text{sample distribution} & \to & \text{posterior distribution} \\[1.5ex]
 f(p) & \times & f(s \mid p) & \propto & f(p \mid s) \\
 & & & & \qquad\text{\bluealert{\play\ Bayes' Rule}} \\}
%\uncover<3->{\downarrow & & \downarrow & & \hspace*{3ex} \downarrow \\
\uncover<4->{\text{Beta prior}}   & & \uncover<3->{\text{Binomial}}         & & \uncover<5->{\text{Beta posterior}} \\
\uncover<4->{}                    & & \uncover<3->{\text{distribution}}     & & \uncover<5->{\qquad \text{\bluealert{\play\ conjugacy}}}\\[1ex]
\uncover<4->{p \sim \be(\az,\bz)} & & \uncover<3->{s \mid p \sim \bin(n,p)} & & \uncover<5->{p \mid s \sim \be(\an,\bn)}
\end{array}
\end{align*}
\vspace*{-3ex}
\begin{tikzpicture}
\uncover<6->{%
\node at (0,0) {\parbox{0.99\textwidth}{%
\begin{itemize}
\item conjugate prior makes learning about parameter tractable,\\  %posterior distribution
      just update hyperparameters:\quad $\az \to \an$, $\bz \to \bn$
\item closed form for some inferences: $\E[p\mid s] = \frac{\an}{\an+\bn}$
\end{itemize}}};}
\uncover<3-5>{%
\node at (-0.5,0) {\includegraphics[width=0.25\textwidth]{figs/smallfig-binom}};}
\uncover<4-5>{%
\node at (-4.5,0) {\includegraphics[width=0.25\textwidth]{figs/smallfig-prior}};}
\uncover<5>{%
\node at ( 3.5,0) {\includegraphics[width=0.25\textwidth]{figs/smallfig-posterior}};}
\end{tikzpicture}
\end{frame}

\begin{frame}{Nonparametric Component Reliability}
%nonparametric model for component failure time distribution\\
%example reliability curve on same frame?
\begin{tikzpicture}
\node at (1,3.5) %
{\parbox{\textwidth}{%
\uncover<1->{%
Functioning probability $\ptk$ of \comp{k} for each time $t \in {\cal T} = \{\dot{t}_1, \dot{t}_2, \ldots \}$\\
\quad\play\ discrete component reliability function $R^k(t) = p^k_t, \ t \in {\cal T}$.}}};
\uncover<2>{\node at (0,0) {\includegraphics[width=0.7\textwidth]{figs/discr-rel-0.pdf}};}
\uncover<3>{\node at (0,0) {\includegraphics[width=0.7\textwidth]{figs/discr-rel-1.pdf}};}
\uncover<4>{\node at (0,0) {\includegraphics[width=0.7\textwidth]{figs/discr-rel-3.pdf}};}
\uncover<5>{\node at (0,0) {\includegraphics[width=0.7\textwidth]{figs/discr-rel-bt1.pdf}};}
\node at (1,0) %
{\parbox{\textwidth}{%
%\uncover<6->{%
%Choosing $\ptk$'s directly is hard (also ignores uncertainty in choice)\\}
\uncover<6->{%
\cyanalert{use Bayesian inference to estimate $\ptk$'s:}} %combine information from
\begin{itemize}
\item[\play]<7-> failure times $\vec{t}^k = (t^k_1, \ldots, t^k_{n_k})$ from test data
 \begin{itemize}
 \item[] number of type $k$ components functioning at $t$:\\
 $S^k_t \sim \bin(\ptk, n_k)$
 \end{itemize}
\item[\play]<8-> expert knowledge
 \begin{itemize}
 \item[] Beta parameters for each $k$ and $t$:\\
 $\ptk \sim \be(\nz_{k,t}, \yz_{k,t})$ \hspace{6ex} \uncover<10->{$\nz_{k,t} = \az_{k,t} + \bz_{k,t}$,
                                                                   \quad $\yz_{k,t} = \frac{\az_{k,t}}{\az_{k,t} + \bz_{k,t}} = \E[\ptk]$}
 \end{itemize}
\item[\play]<9-> complete picture
 \begin{itemize}
 \item[] updated Beta parameters for each $k$ and $t$:\\[1ex]
 $\ptk \mid s^k_t \sim \be(\nn_{k,t}, \yn_{k,t})$ %\hspace{3ex} \uncover<12->{$\nn_{k,t} = \nz_{k,t} + n_k$} \\
%\hspace*{15ex} \uncover<12->{$\yn_{k,t} = \frac{\nz_{k,t}}{\nz_{k,t} + n_k} \, \yz_{k,t} + \frac{n_k}{\nz_{k,t} + n} \cdot \frac{s^k_t}{n_k}$}
 \end{itemize}
\end{itemize}}};
\end{tikzpicture}

\end{frame}

\begin{frame}{System Reliability}

%(adapt slide 8 of ESREL talk)\\
%cite risk analysis paper\\
%frame with example graph?
\play\ Closed form for the system reliability via the survival signature:\\[-2ex]
\begin{tikzpicture} % l b r t
[cyanbox/.style={rounded corners, text centered, fill=tuecyan!20, inner sep=5mm},
 cyanrand/.style={rounded corners, text centered, draw=tuecyan!50, inner sep=2mm, very thick, font=\footnotesize},
 type1/.style={rectangle,draw,fill=tuepmsgreen!70,very thick,inner sep=0pt,minimum size=6mm},
 type2/.style={rectangle,draw,fill=tuepmsgreen!70,very thick,inner sep=0pt,minimum size=6mm},
 type3/.style={rectangle,draw,fill=tuepmsgreen!70,very thick,inner sep=0pt,minimum size=6mm},
 type1a/.style={rectangle,draw,fill=tueyellow!70,very thick,inner sep=0pt,minimum size=6mm},
 hv path/.style={thick, to path={-| (\tikztotarget)}},
 vh path/.style={thick, to path={|- (\tikztotarget)}},
 pfeil/.style={-latex', line width=1mm, color=tuered, shorten <=1mm}]
\uncover<1->{\node at (0,0) %
     {\parbox{\textwidth}{\begin{multline*}
       P\left(\Tsys > t \mid \{\nz_{k,t},\yz_{k,t},\vec{t}^k\}^{1:K}\right)\\
       = \sum_{l_1=0}^{m_1} \cdots \sum_{l_K=0}^{m_K} \Phi(l_1,\ldots,l_K)
         \prod_{k=1}^K P(C^k_t = l_k\mid\nz_{k,t},\yz_{k,t},\vec{t}^k)\end{multline*}}};}
\uncover<2->{\node[cyanrand] (survsign) at (-3.1,-3.6)%
     {\parbox[c]{0.51\textwidth}{Survival signature $\Phi(l_1,\ldots,l_K)$\\ \parencite{2012:survsign}\\
       $= P(\text{system functions} \mid \{l_k \text{ \comp{k}'s function}\}^{1:K})$\\[1ex]
      \begin{tabular}{ccc|l}
      $l_1$ & $l_2$ & $l_3$ & $\Phi$ \\
      \hline
      0 & 0 & 1 & 0 \\
      1 & 0 & 1 & 0 \\
      2 & 0 & 1 & 1/3 \\
      3 & 0 & 1 & 1 \\
      4 & 0 & 1 & 1 %\rule[-2.2ex]{0ex}{1ex}%\phantom{\vdots}
      \end{tabular} \hspace{1ex}
      \begin{tabular}{ccc|l}
      $l_1$ & $l_2$ & $l_3$ & $\Phi$ \\
      \hline
      0 & 1 & 1 & 0 \\
      1 & 1 & 1 & 0 \\
      2 & 1 & 1 & 2/3 \\
      3 & 1 & 1 & 1 \\
      4 & 1 & 1 & 1 
%      \vdots & \vdots & \vdots & \vdots
      \end{tabular} } };
\draw [pfeil] (survsign.north) to [out=50,in=260] (0.3,-0.7);}
\uncover<3-9>{%
\begin{scope}[xshift=1cm,yshift=-3.5cm]
\begin{scope}[scale=0.75]
\uncover<3,5,7,9>{\node[type1 ] (T1-2) at ( 1.2, 1.1) {1};}
\uncover<4,6,8>  {\node[type1a] (T1-2) at ( 1.2, 1.1) {1};}
\uncover<3,4,7,8>{\node[type1 ] (T1-3) at ( 1.2,-1.1) {1};}
\uncover<5,6,9>  {\node[type1a] (T1-3) at ( 1.2,-1.1) {1};}
\uncover<3,5,6,8>{\node[type1 ] (T1-5) at ( 2.8, 1.1) {1};}
\uncover<4,7,9>  {\node[type1a] (T1-5) at ( 2.8, 1.1) {1};}
\uncover<3,4,6,9>{\node[type1 ] (T1-6) at ( 2.8,-1.1) {1};}
\uncover<5,7,8>  {\node[type1a] (T1-6) at ( 2.8,-1.1) {1};}
\node[type2] (T2)   at ( 2.0, 0)   {2};
\node[type3] (T3)   at ( 4.3, 0)   {3};
\coordinate (start) at (0  ,0);
\coordinate (end)   at (4.9,0);
\coordinate (bista) at (0.4,0);
\coordinate (biend) at (3.6,0);
\path (bista)     edge[hv path] (start)
                  edge[vh path] (T1-2.west)
                  edge[vh path] (T1-3.west)
      (T1-2.east) edge[hv path] (T2.north)
      (T1-3.east) edge[hv path] (T2.south)
      (T2.north)  edge[vh path] (T1-5.west)
      (T2.south)  edge[vh path] (T1-6.west)
      (biend)     edge[vh path] (T1-5.east)
                  edge[vh path] (T1-6.east)
                  edge[hv path] (T3.west)
      (T3.east)   edge[hv path] (end);
\end{scope}
\end{scope}}
\uncover<10>{\node[cyanrand] (postpred) at (3,-3.6)%
     {\parbox[c]{0.41\textwidth}{Posterior predictive probability that\\ in a new system, $l_k$ of the $m_k$ \comp{k}'s\\
                                function at time $t$:\\[1ex] \hspace*{1ex}
                                $\binom{m_k}{l_k} \int [P(T <   t \mid \ptk)]^{l_k}$ \\ \hspace*{5.6ex}
                                                        $[P(T \ge t \mid \ptk)]^{m_k-l_k}$ \\ \hspace*{6ex}
%                                $f(\ptk \mid \az_{k,t},\bz_{k,t},\vec{t}^k)\, d\ptk $\\[1ex] \hspace*{5ex}
%                                $= \binom{m_k}{l_k} \frac{B(      \az_{k,t} + s^k_t,             \bz_{k,t} + n_k - s^k_t)}%
%                                                         {B(l_k + \az_{k,t} + s^k_t, m_k - l_k + \bz_{k,t} + n_k - s^k_t)}$}};
                                $f(\ptk \mid \nz_{k,t},\yz_{k,t},\vec{t}^k)\, d\ptk $\\[1ex] %\hspace*{5ex}
(integral can be solved analytically)}};
\draw [pfeil] (postpred.north) to [out=90,in=270] (3.0,-0.7);}
\end{tikzpicture}

\end{frame}

\begin{frame}{System Reliability: Example}

\includegraphics[width=\textwidth]{figs/sysrelex}\\

\end{frame}



\begin{frame}{Vague Knowledge \& Prior-Data Conflict}

\begin{itemize}[<+->]
\item Choosing all these Beta parameters is hard \ldots\\
How to model partial and vague expert knowledge?
\item What if expert information and data tell different stories?\\
How is uncertainty about $\Rsys(t)$ expressed?
\item[\play] \textbf{Add \alert{imprecision} as new modelling dimension:\\
\alert{Sets of priors} model uncertainty in probability statements,\\
allow to better model partial or vague information on $\ptk$
and highlight prior-data conflict.}
\item Separate uncertainty \emph{whithin the model} (reliability statements)\\
% ie what do we expect to see if components follow a Weibull(3.5,2) 
from uncertainty \emph{about the model} (which parameters).
\item \textcite{2009:WalterAugustin}, \textcite{2013:diss-gw}:\\
vary $(\nz, \yz)$ in a set $\PZ = [\nzl, \nzu] \times [\yzl, \yzu]$\\
\uncover<6->{\quad\play\ easy elicitation, tractability \& prior-data conflict sensitivity\\}
\uncover<7->{\quad\play\ min and max $\Rsys(t)$ over $\PZ$ analytical in most cases!}
\end{itemize}
%$\min_{\PZ} P\left(\Tsys > t \mid \{\nz_{k,t},\yz_{k,t},\vec{t}^k\}^{1:K}\right)$

\end{frame}

\iffalse
\begin{frame}{Prior-Data Conflict}
\uncover<1->{What if expert information and data tell different stories?\\[1ex]}
\uncover<2->{\play\ reparametrization helps to understand effect of prior-data conflict:\\}
\begin{tikzpicture}
[pfeil/.style={-latex', line width=1mm, color=tuered, shorten <=1mm},
 cyanrand/.style={rounded corners, text centered, draw=tuecyan!50, inner sep=1mm, line width=0.7mm},
 redbrace/.style={draw=tuered, decoration=brace, decorate, line width=0.8mm},
 redbox/.style={text centered, draw=tuered, inner sep=1.5mm, line width=1mm}]
\uncover<2->{%
\node at (0,0) {\parbox[c]{\textwidth}{%
\begin{align*}
\nz &= \az + \bz\,,
&
\yz &= \frac{\az}{\az + \bz}\,, \qquad \text{where}\\[1ex]
\nn &= \nz + n\,, 
&
\yn &=  \frac{\nz}{\nz + n} \, \yz + \frac{n}{\nz + n} \cdot \frac{s}{n}
\end{align*}
}};}
\uncover<3->{%
\node[cyanrand] (yz) at (-0.8,-2.4) {$\yz = \E[p]$};
\draw [pfeil] (yz.north east) to [out= 40,in=260] (1.90,-1.05);
\draw [pfeil] (yz.north)      to [out=110,in=260] (-0.5, 0.35);}
\uncover<4->{%
\node[cyanrand] (yn) at (1.7,-2.4) {$\yn = \E[p\mid\vec{t}]$};
\draw [pfeil] (yn.north) to [out=90,in=300] (-0.3,-0.95);}
\uncover<5->{%
\node[cyanrand] (ml) at ( 4.3,-2.4) {\parbox[c]{8.3ex}{fraction in\\ \hspace*{0.2ex} test data}};
\draw [pfeil] (ml.north)      to [out=90,in=260] (4.4,-1.25);}
\uncover<6->{%
\node[cyanrand] (nz) at (-3.9,-2.4) {$\nz =$ pseudocounts};
\draw [pfeil] (nz.north)      to [out=80,in=250] (-3.55,-0.95);
\draw [pfeil] (nz.north west) to [out=95,in=240] (-4.6, 0.4);}
\uncover<7->{%
\node[redbox] (wavg) at (0,-3.5) {$\E[p\mid\vec{t}]$ is a weighted average of $\E[p]$ and $s/n$!};}
\end{tikzpicture}
\end{frame}

%\begin{frame}{Prior-Data Conflict: Example}
%\begin{tikzpicture}
%\uncover<1>{\node at (0,0) {\includegraphics[width=0.7\textwidth]{figs/discr-rel-0.pdf}};}
%\end{tikzpicture}
%\end{frame}


\begin{frame}{***pdc etc, 6 slides}
(adapt slides 3--6 of ESREL talk)\\
vague, incomplete prior info, pdc\\
reparametrization\\
Beta example\\
imprecision\\
sets of priors\\
sets of Betas example?
\end{frame}

\begin{frame}{System Reliability Bounds}
(adapt slide 9 of ESREL talk)\\
mostly analytically (via theorems)
\end{frame}
\fi


\begin{frame}{System Reliability Bounds}
\begin{tikzpicture}
[typeM/.style={rectangle,draw,fill=tuepmsgreen!70,thick,inner sep=0pt,minimum size=5mm,font=\footnotesize},
 typeC/.style={rectangle,draw,fill=tuepmsgreen!70,thick,inner sep=0pt,minimum size=5mm,font=\footnotesize},
 typeP/.style={rectangle,draw,fill=tuepmsgreen!70,thick,inner sep=0pt,minimum size=5mm,font=\footnotesize},
 typeH/.style={rectangle,draw,fill=tuepmsgreen!70,thick,inner sep=0pt,minimum size=5mm,font=\footnotesize},
 cross/.style={cross out,draw=red,very thick,minimum width=7mm, minimum height=5mm},
 hv path/.style={thick, to path={-| (\tikztotarget)}},
 vh path/.style={thick, to path={|- (\tikztotarget)}}]
\node at (0,0) {\includegraphics[width=\textwidth]{figs/brakingsystem-mresday16.pdf}};
\begin{scope}[scale=0.90,xshift=3.4cm,yshift=-1.1cm]
\node[typeM] (M)    at ( 0  , 0  ) {M};
\node[typeC] (C1)   at ( 1  , 1.5) {C1};
\node[typeC] (C2)   at ( 1  , 0.5) {C2};
\node[typeC] (C3)   at ( 1  ,-0.5) {C3};
\node[typeC] (C4)   at ( 1  ,-1.5) {C4};
\node[typeP] (P1)   at ( 2  , 1.5) {P1};
\node[typeP] (P2)   at ( 2  , 0.5) {P2};
\node[typeP] (P3)   at ( 2  ,-0.5) {P3};
\node[typeP] (P4)   at ( 2  ,-1.5) {P4};
\node[typeH] (H)    at ( 0  ,-1  ) {H};
\coordinate (start)  at (-0.7, 0);
\coordinate (startC) at ( 0.5, 0);
\coordinate (startH) at (-0.4, 0);
\coordinate (Hhop1)  at ( 0.4,-1);
\coordinate (Hhop2)  at ( 0.6,-1);
\coordinate (endP)   at ( 2.5, 0);
\coordinate (end)    at ( 2.8, 0);
\path (start)     edge[hv path] (M.west)
      (M.east)    edge[hv path] (startC)
      (startC)    edge[vh path] (C1.west)
                  edge[vh path] (C2.west)
                  edge[vh path] (C3.west)
                  edge[vh path] (C4.west)
      (C1.east)   edge[hv path] (P1.west)
      (C2.east)   edge[hv path] (P2.west)
      (C3.east)   edge[hv path] (P3.west)
      (C4.east)   edge[hv path] (P4.west)
      (endP)      edge[vh path] (P1.east)
                  edge[vh path] (P2.east)
                  edge[vh path] (P3.east)
                  edge[vh path] (P4.east)
                  edge[hv path] (end)
      (startH)    edge[vh path] (H.west)
      (H.east)    edge[hv path] (Hhop1)
      (Hhop1)     edge[thick,out=90,in=90] (Hhop2)
      (Hhop2)     edge[hv path] (P3.south)
                  edge[hv path] (P4.north);
\end{scope}
\end{tikzpicture}
\end{frame}

\begin{frame}{Summary \& Outlook}

%(adapt slide 11 of ESREL talk)\\
%case study\\
%block policy with minimal repair?\\
%(nonparam hazard rate $h^k(t_j) = (p^k_{t_j} - p^k_{t_{j+1}}) / p^k_{t_j}$)
\uncover<1->{%
\textbf{Summary:}
\begin{itemize}
\item[\play] Nonparametric modeling of component reliability curves
\item[\play] Bayesian model combining expert knowledge and test data
\item[\play] Set of system reliability functions %appropriately
reflects uncertainties from limited data, vague expert information, and prior-data conflict
\item[\play] Easy-to-use implementation in \textbf{R} package \texttt{ReliabilityTheory} \parencite{2016:aslett-RT}
\end{itemize}}
\uncover<2->{%
\textbf{Next steps:}
\begin{itemize}
\item[\play] Allow right-censored observations (RUL estimation)
\item[\play] Allow dependence between components\\ (common-cause failure, \ldots) %or more explicit dependence
\item[\play] Use for system design (where to put extra redundancy?)
\item[\play] Use for maintenance planning
\end{itemize}}
\end{frame}

\nocite{2015:bayessurvsign, 2016:bayessurvsignsets}

\begin{frame}{References}
\renewcommand*{\bibfont}{\footnotesize}
%\begin{frame}[allowframebreaks]{References}
\printbibliography[heading=none,fontsize=\scriptsize]
\end{frame}

\end{document}
